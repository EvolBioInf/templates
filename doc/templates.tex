\documentclass[a4paper, english]{article}
\usepackage{graphics,eurosym,latexsym}
\usepackage{listings}
\usepackage{pst-all}
\usepackage{algorithmic,algorithm}
\lstset{columns=fixed,basicstyle=\ttfamily,numbers=left,numberstyle=\tiny,stepnumber=5,breaklines=true}
\usepackage{times}
\usepackage{babel}
\usepackage[useregional]{datetime2}
\usepackage[round]{natbib}
\bibliographystyle{plainnat}
\oddsidemargin=0cm
\evensidemargin=0cm
\textwidth=16cm
\textheight=23cm
\begin{document}
\title{\texttt{templates} \input{version}: A Package for Generating
  Code Templates}
\author{Bernhard Haubold\\\small Max-Planck-Institute for Evolutionary
  Biology, Pl\"on, Germany}
\date{\input{date}}
\maketitle
\section{Introduction} 
Programming of any kind often starts from a template rather than from
scratch. The package \texttt{templates} consists of programs for
generating such templates. It currently contains two programs,
\texttt{makeCfile} and \texttt{makeCproject} to facilitate programming
in C. I plan to extend this to other languages. \texttt{MakeCfile}
generates single files with a C-style header. \texttt{MakeCproject}
generates a project template complete with hooks for documentation,
testing, and \texttt{git} integration.

\section{Getting Started}
\texttt{Templates} is written in C on a computer running Linux.
Please contact \texttt{haubold@evolbio.mpg.de} if there are any problems.
\begin{itemize}
\item Obtain the package
\begin{verbatim}
git clone https://www.github.com/evolbioinf/templates
\end{verbatim}
\item Change into the directory just downloaded
\begin{verbatim}
cd templates
\end{verbatim}
and make the programs
\begin{verbatim}
make
\end{verbatim}
\item Test the package
\begin{verbatim}
make test
\end{verbatim}
\item The executables are now located in the directory
  \texttt{build}. Place them into your \texttt{PATH}.
\item  Make the documentation
\begin{verbatim}
make doc
\end{verbatim}
This command calls the typesetting program latex, so make sure it is
installed before making the documentation. The typeset documentation
is located in
\begin{verbatim}
doc/templates.pdf
\end{verbatim}
\end{itemize}

\section{Tutorial}
\subsection{\texttt{MakeCfile}}
\begin{itemize}
\item List the options
\begin{verbatim}
makeCfile -h
\end{verbatim}
\item Generate a new file
\begin{verbatim}
makeCfile -a "My Name" foo.c
\end{verbatim}
The \texttt{-a} option specifying the \emph{author} is mandatory.
\item Look at the new file
\begin{verbatim}
cat foo.c
\end{verbatim}
\item If you repeat the last command, \texttt{makeCfile} refuses to
  overwrite the existing \texttt{foo.c}.
\item By default, the width of the header in a freshly minted file is
  equal to the length of its longest line in that header. If that is
  less than the length of the top line, the header length jumps to 70,
  for example
\begin{verbatim}
./makeCfile -D D -d d -a A longfilename.c
\end{verbatim}
where \texttt{-D} specifies the date (by default this is looked up
automatically) and \texttt{-d} the description.
\item This default behavior can be overridden by directly setting the header
  width using the option \texttt{-w}
\begin{verbatim}
rm longfilename.c
./makeCfile -D D -d d -a A -w 5 longfilename.c
\end{verbatim}
\end{itemize}

\subsection{\texttt{MakeCproject}}
\begin{itemize}
\item List options
\begin{verbatim}
makeCproject -h
\end{verbatim}
\item Generate a project
\begin{verbatim}
makeCproject proj
\end{verbatim}
\item Change into the new directory
\begin{verbatim}
cd proj
\end{verbatim}
\item List its contents
\begin{verbatim}
ls
common data  doc  Makefile  README.md  scripts  src
\end{verbatim}
Each of these files and directories contain typical code examples for
parts of a coding project:
\begin{itemize}
  \item The directory \texttt{common} contains the files \texttt{eprintf.c},
    \texttt{eprintf.h}, and \texttt{Makefile}. The files
    \texttt{eprintf.*} contain code for safely allocating memory taken
    from  \citep[ch. 4]{ker99:pra}. Such external code is usually
    incorporated into a library and hence the \texttt{Makefile}
    contains commands for library construction executed later when we
    make the entire project.
  \item The directory \texttt{data} contains the file
    \texttt{proj.out}, which is later used for testing the project.
  \item The directory \texttt{doc} contains \LaTeX{} code for
    typesetting the documentation.
  \item The file \texttt{Makefile} is used for making the project.
  \item The file \texttt{README.md} contains a brief description of the
    program typeset in \emph{markdown}.
  \item The directory \texttt{scripts} contains scripts for building
    and testing the project.
  \item The directory \texttt{src} contains the sources for a program
    called \texttt{proj}.
\end{itemize}
\item Make the project
\begin{verbatim}
make
\end{verbatim}
This executes the instructions in the \texttt{Makefile} and first generates
the library
\begin{verbatim}
common/libcommon.a
\end{verbatim}
and then the program
\begin{verbatim}
src/proj
\end{verbatim}
which uses the library. The \texttt{make} command also generates a new
directory, \texttt{build}, and places a copy of \texttt{proj} into it.
\item Run the new program
\begin{verbatim}
./build/proj 
Test output.
\end{verbatim}
\item List its options
\begin{verbatim}
./build/proj -h
\end{verbatim}
\item Run the program
\begin{verbatim}
./build/proj -i 2
\end{verbatim}
\item Try the automated testing mechanism
\begin{verbatim}
make test
\end{verbatim}
The important output line begins with
\begin{verbatim}
Test(proj)
\end{verbatim}
followed by \texttt{pass} or \texttt{fail} (it should be
\texttt{pass}). The script generating this output is
\begin{verbatim}
scripts/proj.sh
\end{verbatim}
\item Print the version information
\begin{verbatim}
./build/proj -v
\end{verbatim}
This is generated from the latest \texttt{git}\footnote{Common
  software for versioning source files.} tag and commit. To see
how this works, first take a look at the \texttt{git} tags for the
project
\begin{verbatim}
git tag
v0.1
\end{verbatim}
Now change the project slightly and \textit{commit} the change. 
Open the file
\begin{verbatim}
src/proj.c
\end{verbatim}
and change the line
\begin{verbatim}
printf("Test output.\n");
\end{verbatim}
to, say,
\begin{verbatim}
printf("Test out.\n");
\end{verbatim}
Add the change to the \texttt{git} repository and commit it
\begin{verbatim}
git add src/proj.c 
git commit -m "Test commit"
\end{verbatim}
\item Make the project again
\begin{verbatim}
make clean
make
\end{verbatim}
and take a look at the version
\begin{verbatim}
./build/proj -v
\end{verbatim}
Note the \texttt{-1} that now follows the tag number. This is the
commit-counter for that tag. To increment the commit-counter, change 
\begin{verbatim}
printf("Test out.\n");
\end{verbatim}
back to 
\begin{verbatim}
printf("Test output.\n");
\end{verbatim}
and repeat the add/commit cycle.
\item The tag is also passed on to the documentation, which helps
  keeping it synchronized with the program. The command
\begin{verbatim}
make doc
\end{verbatim}
generates the documentation file
\begin{verbatim}
doc/proj.pdf
\end{verbatim}
which contains the latest tag number in the title. Note that the
commit we just executed is not passed on. Many code changes do not
warrant a change in the documentation, hence the commit count is
ignored in the documentation. Conversely, whenever the documentation
is substantially revised, a new \texttt{git} tag should be set. The
date of the documentation is the day the tag was set.
\end{itemize}

\section{Change Log}
Please use
\begin{verbatim}
git log
\end{verbatim}
to list the change history of this program.

\bibliography{ref}
\end{document}

